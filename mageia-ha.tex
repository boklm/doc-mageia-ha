\documentclass[a4paper,12pt]{article}

\usepackage[english]{babel}
\usepackage{amsfonts}
\usepackage{graphicx}
\usepackage[techreport]{hupstream_cover}
\usepackage{fancyhdr}
\pagestyle{fancy}
\usepackage{hyperref}

\def \doctitle {Mageia High Availability Cluster}
\renewcommand{\footrulewidth}{2mm}
\fancyhf{}
\rhead{\doctitle}
\cfoot{\thepage}

\HUPTitle{\doctitle}
\HUPVersion{0.1}
\HUPLastEditor{boklm}
\HUPLastDate{\today}
\HUPCCBYSA

\HUPRevision{0.1}{2012/05/07}{boklm}{Start document}

\begin{document}
\tableofcontents
\cleardoublepage

\section{Introduction}

\cleardoublepage
\section{About this document}
\subsection{License}

This document is available under the \emph{Creative Commons
Attribution-ShareAlike 3.0} license. 

\hfill \includegraphics[width=3cm]{by-sa.png}

You are free :
\begin{itemize}
\item \textbf{to share} -- to copy, distribute and transmit the work
\item \textbf{to Remix} -- to adapt the work
\item to make commercial use of the work
\end{itemize}

Under the following conditions :
\begin{itemize}
\item \textbf{Attribution} -- You must attribute the work.
\item \textbf{Share Alike} -- If you alter, transform, or build upon
      this work, you may distribute the resulting work only under the same or
      similar license to this one.
\end{itemize}

This is only a summary of the license, you can find more information at
this URL :
\url{http://creativecommons.org/licenses/by-sa/3.0/}

\cleardoublepage
\subsection{Mageia}
\hfill \includegraphics[width=3cm]{mageia-logo.png}

\emph{Mageia} is a Linux distribution started in September 2010 as a
\emph{Mandriva} fork. It is a community distribution supported by a
not-for-profit organisation.

The distribution is based on the \emph{RPM} package manager and includes
various packages for desktop or server use.

The second version of the distribution, to be released in May 2012,
provides tools to setup an high availability cluster.

You can find more informations about \emph{Mageia} on \url{http://www.mageia.org/}.

\cleardoublepage
\subsection{Hupstream}
\hfill \includegraphics[width=5cm]{hupstream_logo.png}

\emph{Hupstream} is an Open Source engineering Studio, based in Paris
and Nantes, in France.

The \emph{Hupstream} team is actively contributing to \emph{Mageia} as
developers, board members and co-founders.

Services provided by \emph{Hupstream} includes :
\begin{itemize}
\item Custom-made Linux distributions
\item RPM or Debian build systems
\item High Availability clusters
\item Hardware integration and qualification with Linux systems (x86/ARM/MIPS)
\end{itemize}

You can find more informations about \emph{Hupstream} on \url{http://www.hupstream.com/}.

\cleardoublepage
\section{High Availability tools in Mageia}
\subsection{DRBD}
The \emph{Distributed Replicated Block Device (DRBD)} is a tool that
allows mirroring of block devices between hosts, using the network. It
allows to transparently mirror data in real time between different hosts.

\emph{DRBD} is very useful to create a fail-over cluster, with one
active node and one passive node. The data manipulated on the first
node is mirrored to the second node using \emph{DRBD}. In case of
failure on the first node, all the data is immediatly available on the
second node which can become the new active node.

\emph{DRBD} is implemented as a kernel module, and some user space tools.
The \emph{DRBD} kernel module is included in the mainline kernel since
version \emph{2.6.33}. In \emph{Mageia}, this module is included with
the server kernel which can be installed using the
\emph{kernel-server-latest} package.

The user space administration tools can be installed with the \emph{drbd-utils}
package. The utilities provided by this package are :
\begin{description}
\item[drbd-overview] to look at \emph{DRBD}'s status,
\item[drbdadm] the high-level administration tool,
\item[drbdsetup] the low lever setup tool,
\item[drbdmeta] the meta data management tool.
\end{description}

You can find more information about \emph{DRBD} on \url{http://www.drbd.org/}.

\subsection{CoroSync}
The \emph{Corosync Cluster Engine} is a \emph{Group Communication System}.
It provides an infrastructure allowing clients to know about the presence
or disappearance of processes on other machines, and exchange of messages.

It is a replacement for the \emph{heartbeat} software used in older
Linux High Availability clusters.

In order to be useful, it needs to be combined with a \emph{cluster
resource manager (CRM)} such as \emph{Pacemaker}.

In \emph{Mageia}, it is available in the \emph{corosync} package.

You can find more information about \emph{CoroSync} on
\url{http://www.corosync.org/}.

\subsection{Pacemaker}

\cleardoublepage
\section{Types of high availability clusters}
TODO

\cleardoublepage
\section{Installing an high availability cluster with Mageia}
TODO

\cleardoublepage
\end{document}
